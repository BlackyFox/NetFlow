\section{Introduction}
\subsection{Le projet ALCASAR}

\begin{wrapfigure}{r}{0.30\textwidth}
  \vspace{-20pt}
  \begin{center}
    \includegraphics[width=0.25\textwidth]{img/Alcasar-logo.png}
  \end{center}
  \vspace{-15pt}
  \caption{Logo d'ALCASAR}
  \vspace{-10pt}
\end{wrapfigure}
Le projet ALCASAR est né suite à une demande d'une entité gouvernementale voulant assurer la traçabilité et l'imputabilité des connexions de tous les utilisateurs connectés à leur réseau.\\
Après une rapide étude de marché, les responsables de projet se sont rendu compte qu'il n'existait pas de solution libre de droits permettant de remplir les demandes précises de l'entité : intercepter, authentifier, filtrer et imputer l'accès aux utilisateurs à internet.\\
Ainsi, les chefs du projet ont alors décidé de réaliser eux-mêmes un contrôleur d'accès pouvant répondre aux besoins du projet. C'est ainsi que le projet ALCASAR à vu le jour.\\\par
Afin de pouvoir réaliser ces fonctions d'interception ou de filtrage, ALCASAR va embarquer plusieurs outils d'analyse de flux. Parmi ces outils, nous allons pouvoir retrouver NetFlow.

\subsection{NetFlow}

NetFlow est un protocole qui a été conçu par Cisco Systems. Il va permettre de réaliser une surveillance des réseaux en collectant des informations sur les flux IP. Ce produit va permettre à un administrateur réseau de déterminer différentes informations telles que la source et la destination du trafic, le type de service utilisé ou encore la cause de nœuds de congestion.

\subsection{IPFIX}
A voir : \url{http://www.bradreese.com/blog/netflow-vs-ipfix-exporter.htm}