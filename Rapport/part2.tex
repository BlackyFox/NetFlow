\section{NetFlow}
\subsection{Indtoduction}
NetFlow est un protocole développé par la société Cisco dans le but de faire une analyse des trames passant au travers de leurs équipements réseau. Ce projet ayant une très grande popularité lors de sa sortie, le protocole devint alors accessible à tous.\\
Ainsi, ici, afin d'utiliser le protocole NetFlow, nous pouvons retrouver plusieurs parties :
\subsubsection{La sonde}
La mise en place du protocole NetFlow est ici faite grâce un module noyau, \textit{ipt\_NETFLOW}.\\
TODO\\
Cette sonde va envoyer toutes ses données à une adresse IP et un port bien spécifique (ici, 127.0.0.1:2055).
\subsubsection{Le collecteur}
Afin de récolter toutes les données fournies par la sonde NetFlow, nous devons aussi avoir un collecteur. Son rôle est d'écouter à l'adresse sur laquelle la sonde NetFlow envoie ses données. Il va ainsi réaliser un fichier au format \enquote{\textit{NetFlow}} toutes les 5 minutes.\\
Ici, le rôle du collecteur est réalisé par le daemon \textit{Nfcapd}.
\subsubsection{L'interpréteur}
Cependant, les données fournies par Nfcapd au format \enquote{\textit{NetFlow}} ne sont pas lisible. Afin de palier à cela, il est possible d'utiliser un interpréteur. Dans notre cas, ce rôle est joué par \textit{Nfdump}.\\
Il est alors possible d'avoir le contenu des captures de la sonde NetFlow à n'importe quel moment.
\subsubsection{Le grapheur}
Cependant, au sein de l'interface de gestion d'ALCASAR (ACC\footnote{ALCASAR Control Center}), il est intéressant de pouvoir retrouver de manière graphique les résultats des captures de la sonde NetFlow.\\
Afin de réaliser cela, nous utilisons le grapheur Nfsen. Ce dernier va récupérer les données fournies par le collecteur afin de réaliser des graphes représentatifs de l'état des connexion (charges, nombre de connexion, serveurs les plus demandés, etc\ldots).

\subsection{Mise à jour}
Afin de mettre toute la partie NetFlow à jour au sein d'ALCASAR, nous avons tout d'abord chercher quelles étaient les versions installées sur la dernière version du NAC.
\begin{table}[H]
    \centering
    \begin{tabularx}{\textwidth}{|l|c|c|}
	\hline
	& \textbf{Version au sein d'ALCASAR} & \textbf{Dernière version}\\\toprule
	\textbf{ipt\_NETFLOW} & 1.7.2 & 2.2-36\\\hline
 	\textbf{Nfdump} & 1.6.9 & 1.6.14\\\hline
 	\textbf{Nfsen} & 1.3.7 & 1.3.7\\\hline
 	\textbf{Nfcapd} & 1.6.9 & 1.6.14\\\hline
    \end{tabularx}
    \caption{Comparaison des versions des modules}
    \label{tab:1}
\end{table}
Comme vous pouvons le voir sur le tableau \ref{tab:1}, nous devons mettre à jour la sonde ipt\_NETFLOW, Nfdump ainsi ue Nfcapd.\\
Actuellement, Nfcapd est inclut au sein de Nfdump. Ainsi, nous devons mettre à jour la sonde et l'interpréteur.\\
Pour cela, nous avons récupéré les sources des deux paquets sur leurs sources officielles :
\begin{description}
 \item[ipt\_NETFLOW] \url{https://github.com/aabc/ipt-netflow}
 \item[Nfdump] \url{https://github.com/phaag/nfdump}
\end{description}

\subsubsection{ipt\_NETFLOW}
Afin d'installer la sonde sur notre poste ALCASAR, nous devons ajouté ce module au noyau. Cependant, le noyau d'ALCASAR est déjà configuré pour la mise en place d'un module NetFlow.\\
ipt\_NETFLOW étant un module supplémentaire d'iptables, il faut s'assurer que ce dernier comporte bien toutes les librairies nécessaires. Pour ce faire, il est recommander de télécharger et installer iptables depuis les sources\footnote{\url{ftp://ftp.netfilter.org/pub/iptables/}}. Il faut cependant faire attention à bien prendre la même version que celle installé de base dans le noyau de la machine. Dans notre cas, pour Mageia 5, la version d'iptables en question est la 1.4.21\footnote{\url{ftp://ftp.netfilter.org/pub/iptables/iptables-1.4.21.tar.bz2}}.\\
Une fois le paquet téléchargé, nous pouvons décompresser l'archive et l'installer :
\begin{lstlisting}[style=custombash, language=bash]
cd iptables-1.4.21
./configure --prefix=/usr/
make
make install 
\end{lstlisting}
Dans certains cas, il est possible de vouloir utiliser un système de supervision ou d'hypervision dans le parc informatique où est installé ALCASAR. C'est dans cette optique que les développeurs d'ipt\_NETFLOW ont pensé à intégrer le module net-snmp. \textbf{\textit{DESCRIPTION DU MODULE}}.\\
L'installation de ce module n'est pas obligatoire pour le bon fonctionnement de NetFlow. Nous avons cependant préféré tester son installation afin de fournir une installation complète de NetFlow.\\
Il est possible de télécharger la dernière version de ce module sur le site officiel de net-snmp\footnote{\url{www.net-snmp.org}}. Dans notre cas, il s'agit de la version 5.3.7.\\
Il faut aussi noter que pour que l'installation de ce module fonctionne, il est nécessaire d'avoir des \textit{header} pour perl. Ces fichiers n'étant pas présents lors de nos tests sur Mageia 5, nous avons du installer le paquet \textit{perl-devel}.\\
Une fois téléchargé, nous pouvons l'installer :
\begin{lstlisting}[language=bash,style=custombash]
Erreur latex ici
\end{lstlisting}
Une fois ce module installé correctement, il est enfin possible de réaliser l'installation d'ipt\_NETFLOW. La version utilisée est la dernière version disponible du le GitHub du projet (2.2). Git n'étant pas installé de base sur Mageia 5, nous fournissons l'archive comprenant cette version\footnote{ipt-netflow22.tar.gz}.\\
Une fois décompressée, il est possible de l'installer de la manière suivante :
\begin{lstlisting}[language=bash,style=custombash]
 cd ipt-netflow22
 ./configure
 make all install
 depmod
\end{lstlisting}
Une fois l'installation réalisée, nous pouvons redémarrer la machine afin que le script de démarrage d'ALCASAR initialise correctement la sonde NetFlow.